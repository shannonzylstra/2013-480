\documentclass{article}
\usepackage{sagetex}

\usepackage{array}
\usepackage{amsmath,amssymb}
\usepackage{booktabs}
\usepackage{caption}
\usepackage{environ}
\usepackage{float}
\usepackage{enumitem}
\usepackage{graphicx}
\usepackage{hyperref}
\usepackage{multicol}
\usepackage{rotating}
\usepackage{tikz}
\usepackage{url}
\usepackage{xspace}
\usepackage{dsfont}
\usepackage{cancel}
\usepackage{xargs}
\usepackage{color}
\usepackage{tabularx}
\usepackage[usenames,dvipsnames,svgnames,table]{xcolor}

\newcommand{\E}[2][]{\ensuremath{\mathbb{E}_{#1}\left[#2\right]}}
\newcommand{\V}[2][]{\ensuremath{\mathbb{V}_{#1}\left[#2\right]}}

\title{Probability in Sage with \LaTeX{}}
\author{Shannon Rae Zylstra}
\date{May 22-23, 2013}

\begin{document}

\maketitle % typeset it?

\begin{abstract}
The selection of probability distributions and statistical calculations follows the syllabus of the
first actuarial exam, which is published by the Society of Actuaries (SOA) and which constitutes
a sufficiently comprehensive and foundational treatment of probability.
\end{abstract}

\newpage

\vfill
\tableofcontents
\clearpage

\section{Introduction}
For my project, I will define a new class of random variable.
This random variable object will follow a given probability distribution and will allow users to perform various statistical functions.
Supported distributions and functions were chosen from the Exam P/1 Syllabus. \\

\subsection{List of Supported Distributions}
\begin{itemize}
\item{Discrete Distributions}
    \begin{itemize}
     \item Bernoulli
     \item Binomial
     \item Hypergeometric
     \item Negative Binomial
     \item Geometric
     \item Poisson
    \end{itemize}
\item{Continuous Distributions}
    \begin{itemize}
     \item Exponential
     \item Gamma
     \item Normal
     \item Uniform
    \end{itemize}
\end{itemize}
\subsection{List of Supported Functions}
\begin{itemize}
    \item \text{Probability Function}
    \begin{itemize}
        \item \text{Probability Mass Function} $ p_X(x) $
        \item \text{Probability Density Function} $ f_X(x) $
    \end{itemize}
    \item \text{Cumulative Distribution Function} $ F_X(x) $
    \item \text{Expectation} $ \E{X} $
    \item \text{Variance} $ \V{X} $
\end{itemize}

\section{Example}
\subsection{Binomial}
\subsubsection{Notation}
Unfortunately for students, the notation used to represent a given parametric distribution
can vary significantly. To help students avoid the confusion that can arise,
the notation() function will return the LaTeX representation of the most widely
accepted choice of notation for that distribution (with preference given to the
representations that are the most widely used in  actuarial communities). \\
\subsubsection{Probability Function}
\subsubsubsection{Probability Mass Function}
The pmf(x) function will evaulate the pmf of the random variable at x.
\subsubsubsection{Probability Density Function}
The pdf(x) function will evaulate the pdf of the random variable at x.
\subsubsection{Cumulative Distribution Function}
The cdf(x) function will evaulate the cdf of the random variable at x.
\subsubsection{Expectation}
The expectation() function will evaulate the expectation of the random variable.
\subsubsection{Variance}
The variance() function will evaulate the variance of the random variable.

\section{Using Sage}

Let us define a new class:

\begin{sageblock}
    from sage.rings.arith import binomial
    import scipy
    from scipy.stats import binom

    class Binomial():
        def __init__(self, n, p):
            self._n = n
            self._p = p
            self._rv = binom(n, p)
        def n(self):
            return self._n
        def p(self):
            return self._p
        def rv(self):
            return self._rv
        def pmf(self, x):
            rv = self.rv()
            return rv.pmf(x)
        def cdf(self, x):
            rv = self.rv()
            return rv.cdf(x)
        def expectation(self):
            n = self.n()
            p = self.p()
            return binom.expect(args=(n,p))
        def variance(self):
            n = self.n()
            p = self.p()
            return binom.var(n, p)
\end{sageblock}

Now, let us do something cool with it!

\begin{sageblock}
    D = [0,1,2,3,4,5]
    P = [Binomial(5,0.5).pmf(0),
         Binomial(5,0.5).pmf(1),
         Binomial(5,0.5).pmf(2),
         Binomial(5,0.5).pmf(3),
         Binomial(5,0.5).pmf(4),
         Binomial(5,0.5).pmf(5)]
    pmf = zip(D,P)
    n = len(D)
    R = range(n)
    G = Graphics()
    for k in R:
       G += line([(D[k],0),(D[k],P[k])])
    G += points(pmf,size=50)
\end{sageblock}

Here is a plot of the pmf:

$$\sageplot[scale=.25]{plot(G,ymin=0)}$$

\end{document}